\documentclass[12pt]{article}
\usepackage[latin1]{inputenc}
\usepackage{graphicx,subfigure}
\topmargin -1mm
\oddsidemargin -1mm
\evensidemargin -1mm
\textwidth 165mm
\textheight 220mm
\parskip 2mm
\parindent 3mm
%\pagestyle{empty}

\begin{document}
\begin{center}
{\Large \bf Installation of OpenCalphad on Windows using MinGW}

Bo Sundman, \today

\end{center}

There is no automatic installation routine for OC, you must download
and compile the software yourself.  You may also have to install
Fortran compilers and the GNUPLOT software if you do not already have
them.  The OC development team cannot offer you any help for that,
please ask some local experts if you need help.

On Windows you can use a native Fortran compiler like Microsoft
Developer Studio or Intel.  There is no installation guide for that.
Maybe you can use some of the information below.  Usually you have to
pay for these so you may be able to get some help from the vendor if
you have problems.

Most academics use the free MinGW or Cygwin system with several
compilers which emulates (in slightly different ways) a linux computer
on Windows.  This guide is for MinGW.  There is a French guide for
Cygwin which is more elaborate and does not require that you
understand much French.

The descrcipton below applies when installing OC on Windows using
MinGW, the guides available are:
\begin{itemize}
\item Install-OC-Windows-MinGW
\item Installation de OC sous Windows avec Cygwin (in French)
\item Install-OC-Linux
\end{itemize}

Step by step installation:

\begin{itemize}
\item The code is written in the new Fortran standard and requires a
  compiler like GNU Fortran 4.8 or similar.

\item If you have not already installed MinGW and the Fortran compiler
  you must do that from https://SourgeForge.net or some similar site.
  If you have MinGW but not the Fortran compiler you must add that.
  The MinGW software is free.

\item Rename the file ``linkmake'' to linkmake.cmd so it can be executed.

\item If you have access to several CPUs you can test OC with
  parallelization using Open MP.  In that case you should use the
  linkfile ``linkpara'' below (after renaming it to linkpara.cmd).

\item Open a terminal window.  If you do not know what is a terminal
  window you should ask a local expert.  Keep him or her with you
  until you finished the installation.

\item In the terminal window you may have to use ``cd'' (change
  directory) until you reach the direcory where you unzipped OC.  Then
  exectute the file you just renamed by typing its name.

\item {\bf If you have errors running the linkmake or linkpara command
  files please contact a local expert.}

\item For the graphics you must download and install the free GNUPLOT
  software, you can find that on SoureForge.

  Make sure your PATH includes the directory with the GNUPLOT program.
  If you do not know how to set your PATH ask a local expert.

\item Creating a home directory for OC
  \begin{itemize} 
    \item Create a directory called OCHOME at you home directory,
      usually\\
      ``C:${\backslash}$Users${\backslash}$yourname''.

    \item Create an environment variable for your account called
      OCHOME with the path to your OCHOME directory as value.  If you
      do not know how to create an environment variable please ask a
      local expert.
      
% On my Swedish Windows system I create a environment variable by
% clicking on the Windows icon and then type ``milj�'' and select
% ``Redigera milj�variabler f�r kontot''

      Normally you have to restart your computer to make the
      environment variable available.

    \item Copy the file ochelp.hlp to this directory
    
    \item Later you may also add a macro file on this directory called
      ``start.OCM'' that will be run everytime you start OC.  You may
      also create a direcory called ``databases'' with databases you
      use.  Such databases will be searched if you prefix the database
      name with ``ocdata/'' in the command ``read tdb''

    \item If you want to start the OC program from any directory copy
      also the executable to OCHOME and add the path to OCHOME to your
      \%PATH\%
  \end{itemize}

\item Look in ``after-installation'' for help to use OC.

\end{itemize}

You are welcome to help providing a better installation guide also!

\bigskip

{\large \bf Have fun and help make OC useful!}

\end{document}

